%%
%%  FILE calend1.tex
%%
\def\loadadvanced{\relax}
% Convert from Julian date and time in
% \date to long date (in local time) 
\def\JDTtoL{\advance\date by500
 \advance\date by\timezone\divide\date by1000}
%% Trigonometric functions
\def\sintable#1{\ifcase #1 0\or100\or199
 \or296\or389\or479\or565\or644\or717
 \or783\or841\or891\or932\or964\or985
 \or997\or1000\or992\or974\or946\or909
 \or863\or808\or746\or675\or598\or516
 \or427\or335\or239\or141\or42\or-58
 \or-158\fi}
% Reduces modulo 2\pi (requires positive
% argument theta):
% theta := theta MOD 2\pi, where
% theta = count1*10^(-3)
\def\twopimod{\count2 =\count1
 \divide\count2 by6284 \count3 =1853
 \count4 =6283\multiply\count3  by\count2
 \multiply\count4  by\count2  
 \divide\count3  by10000
 \advance\count3  by\count4  
 \advance\count1  by-\count3}
\newif\ifsign
% v := sin(theta), where
% v = count4*10^(-3);
% theta = count1*10^(-3)
% theta is reduced MOD 2\pi to be
% 0<=theta<2\pi by \TWOPIMOD,
% then linear interpolation is performed
% using \SINTABLE.
\def\Sin{
 \ifnum\count1<0 \signtrue
 \count1=-\count1\else \signfalse\fi
 \loop\ifnum\count1>6284\twopimod\repeat
 \ifnum\count1>3142
 \advance\count1 by-3142
 \ifsign\signfalse\else\signtrue\fi\fi
 \multiply\count1  by10 \count3 =\count1 
 \divide\count3  by1000 \count2 =\count3 
 \multiply\count3  by1000
 \advance\count3  by-\count1  
 \count5 =\sintable{\count2 }
 \count4 =\count5\advance\count2  by1
 \advance\count4  by -\sintable{\count2 }
 \multiply\count4  by\count3  
 \divide\count4  by1000
 \advance\count4  by\count5  
 \ifsign\count4 =-\count4 \fi}
% v := cos(theta), where
% v = count4*10^(-3);
% theta = count1*10^(-3).
% \SIN is evaluated on pi/2-theta.
\def\Cos{\advance\count1  by-1571
 \multiply\count1  by-1\Sin}
% Linear transformation of T giving
% theta := a*T+b, where
% T = count0*10^(-6);
% theta = count1*10^(-3);
% a = #1*10^3+#2+#3*10^(-3);
% b = #4*10^(-3)
\def\lin#1.#2.#3+#4.{\count1 =#3
 \count2 =#2 \count3 =#1 
 \multiply\count1 by\count0
 \multiply\count2 by\count0 
 \multiply\count3 by\count0
 \divide\count1 by1000
 \advance\count1 by\count2
 \divide\count1 by1000
 \advance\count1 by\count3
 \advance\count1 by #4}
% Accumulate value returned by a
% trigonometric function, scaled by
% factor f, into count6:
% ac := ac+f*v, where
% ac = count6*10^(-7)
% v = value of SIN or COS =count4*10^(-3)
% f = #1*10^(-4)
\def\fac#1{\multiply\count4  by #1
 \advance\count6  by\count4}
\def\id{\count4=\count1}  % Identity
